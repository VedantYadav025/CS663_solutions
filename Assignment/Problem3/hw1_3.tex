\documentclass[12pt]{article}
\usepackage{enumitem}
\usepackage{amsmath}

\title{Problem 3: Solution}
\author{Vedant Yadav, Susmit Sarkar, and Atharva Inamdar}
\date{\today}

\begin{document}

\maketitle

\section*{Solution}

Let $I + J = K$, and let $f(x,y)$ and $g(x,y)$ denote the intensities of $I$ and $J$ respectively. When we add the two images, we get
\begin{equation}
    K = f(x,y) + g(x,y)
\end{equation}

Let $p_K(k)$ denote the PMF of the image $K$. This implies that
\begin{align}
    p_K(k) &= P(f(x,y) + g(x,y) = k) \quad \text{for all } (x,y) \text{ in the domain of the image.} \nonumber \\
    &= \boxed{\sum_{r = 1}^{k} p_I(r) \cdot p_J(k - r)}
\end{align}

The above operation is similar to the \textbf{convolution} operator.

\end{document}
